\documentclass{article}
\usepackage{pgfplots}
\pgfplotsset{compat=1.9}
\usepackage{indentfirst}
\usepackage{colortbl}
\usepackage{graphicx}
\DeclareGraphicsExtensions{.pdf,.png,.jpg}
\usepackage[utf8]{inputenc}
\usepackage[english, russian]{babel}
\usepackage[top=2cm, bottom=2cm, left=2cm,right=2cm]{geometry}
\pagestyle{empty}
\textrm

\begin{document}
\centerig {\Large Министерство науки и высшего образования}
\centering{\Large \\Федеральное государственное бюджетное образовательное}
\centering{\Large \\учреждение высшего образования}
\centering{\Large \\``Московский государственный технический университет}
\centering{\Large \\имени Н.Э. Баумана}
\centering{\Large \\(национальный исследовательский университет)''}
\centering{\Large \\(МГТУ им. Н.Э. Баумана)}
\hrule
\vspace{0.5cm}
\begin{figure}[h]
\center
\includegraphics[height=0.35\linewidth]{bmstu-logo.png}
\end{figure}
\begin{center}
	\large	
	\begin{tabular}{c}
		Факультет ``Фундаментальные науки'' \\
		Кафедра ``Высшая математика''		
	\end{tabular}
\end{center}
\vspace{0.5cm}
\begin{center}
	\LARGE \bf	
	\begin{tabular}{c}
		\textsc{Отчёт} \\
		по учебной практике \\
		за 1 семестр 2020---2021 гг.
	\end{tabular}
\end{center}
\vspace{0.5cm}
\begin{center}
	\large
	\begin{tabular}{p{5.3cm}ll}
		\pbox{
			Руководитель практики,\\
			ст. преп. кафедры ФН1} 	& \tline{\it(подпись)} & Кравченко О.В. \\
		\\студент группы ФН1--11 		& \tline{\i(подпись)} & Шалютина Е. Д.
	\end{tabular}
\end{center}
\vfill
\begin{center}
	\large	
	\begin{tabular}{c}
		Москва, \\
		2020 г.
	\end{tabular}
\end{center}

\newpage
\begin{flushleft}

  \textbf{\LARGE Содержание}
\noindent{\section{\textcolor{blue}{\textbf{ Цели и задачи практики\\}}}}
\subsection{\textcolor{blue}{Цели .....................................................................................................3}}
\subsection{\textcolor{blue}{Задачи ..................................................................................................3}}
\subsection{\textcolor{blue}{Индивидуальное задание.....................................................................3\\}}
\indent{\section{\textcolor{blue}{\textbf{Отчет}                                  \\}}}
\indent{\section{\textcolor{blue}{Индивидуальное задание                          }}}
\subsection{\textcolor{blue}{Пределы и непрерывность..................................................................5\\}}
\indent{\textcolor{blue}{\textbf{\Large Список литературы                                       }}}
\end{flushleft}
\newpage
\setcounter{section}{0}
\setcounter{subsection}{0}
\begin{flushleft}
  \section{Цели и задачи практики}
  \subsection{Цели\\}
  \text{-- развитие компетенций, способствующих успешному освоению материала бакалавриата и необходимых в будущей профессиональной деятельности}
  \subsection{Задачи}
  \begin{enumerate}
\setcounter{enumi}{0}
\item Знакомство с программными средствами, необходимыми в будущей профессиональной деятельности.
\item Развитие умения поиска необходимой информации в специальной литературе и других источниках.
\item Развитие навыков составления отчетов и презентации результатов.
\end{enumerate}
\subsection{Индивидуальное задание}
\begin{enumerate}
\setcounter{enumi}{0}
\item Изучить способы отображения математической информации в системе верстки {\LaTeX}.
\item Изучить возможности системы контроля версий Git.
\item Научиться верстать математические тексты, содержащие формулы и графики в системе {\LaTeX}. Для этого выполнить установку свободно распространяемого дистрибутива TeXLive  и оболочки TeXStudio.
\item Оформить в системе {\LaTeX} типовые расчеты на курсе математического анализа согласно своему варианту.
\item Создать аккаунт на онлайн ресурсе GitHub и загрузить исходные tex-файлы и результат компиляции в формате pdf. 
\end{enumerate}
\end{flushleft}
\newpage
\section{Отчет}
\begin{flushleft}
  
Актуальность темы продиктована необходимостью владеть системой верстки {\LaTeX} и средой верстки TeXStudio для отображения текста, формул и графиков. Полученные в ходе практики навыки могут быть применены при написании курсовых проектов и дипломной работы, а также в дальнейшей профессиональной деятельности.\\
\parindent{Система верстки {\LaTeX} содержит большое количество инструментов (пакетов), упрощающих отображение информации в различных сферах инженерной и научной деятеьности.}

\end{flushleft}
\newpage
\begin{flushleft}

\section {Индивидуальное задание}
\subsection{Пределы и непрерывность}
\subsubsection*{\center Задача № 1.}
{\bf Условие.~}

Дана последовательность $\{a_n\} = \frac{2n-2}{3+4n}$  и число $c=\frac{1}{2}$. Доказать, что 
$$\lim\limits_{n\rightarrow\infty}a_n=c,$$

а именно, для каждого сколь угодно малого числа ${\epsilon}>0$ найти наименьшее натуральное число 
$N=N({\epsilon})$ такое, что $|a_n-c|<{\epsilon}$ для всех номеров $n>N({\epsilon})$.
Заполнить таблицу
\begin{center}
	\begin{tabular}{|c|c|c|c|}
		\hline
		$\eps$ &  $0{,}1$ & $0{,}01$ & $0{,}001$ \\
		\hline
		$N(\eps)$ & & & \\
		\hline
	\end{tabular}
\end{center}
{\bf Решение.~}	
Рассмотрим неравенство $a_n-c<{\epsilon},\,\forall{\epsilon}>0$, учитывая выражение для $a_n$ и значение $c$ из условия варианта,
получим
$$
\biggl|\frac{2n-2}{3+4n}-\frac12\biggr| < {\epsilon}.
$$
Неравенство запишем в виде двойного неравентсва и приведём выражение под знаком модуля к общему знаменателю,
получим
$$
-{\epsilon} < \biggl|\frac{7}{(6+8n)}\biggl| < {\epsilon}.
$$
Заметим, что левое неравенство выполнено для любого номера $n\in\mathbb{N}$, поэтому будем рассматривать правое неравенство 
$$
\biggl|\frac{7}{6+8n}\biggl| < {\epsilon}.
$$
Выполнив цепочку преобразований, перепишем неравенство относительно $n$, и учитывая, что $n\in\mathbb{N}$, получим
$$
\begin{array}{c}
|\frac{3}{4}+n| > \frac{7}{8{\epsilon}}, 							\\[20pt]
n > \frac{7}{8{\epsilon}}-\frac34, 		\\[20pt]
N({\epsilon}) = \biggl[\frac{7}{8{\epsilon}}-\frac34\biggl],
\end{array}
$$
где $[\phantom{a}]$ --- целая часть числа.
Заполним таблицу:
\begin{center}
	\begin{tabular}{|c|c|c|c|}
		\hline
		$\eps$ &  $0{,}1$ & $0{,}01$ & $0{,}001$ \\
		\hline
		$N({\epsilon})$ & 8 & 87 & 874 \\
		\hline
	\end{tabular}
\end{center}
\textbf{Проверка:}
$$
\begin{array}{l}
|a_9 - c| = \frac{7}{78} < 0{,}1,			\\[10pt]
|a_{88} - c| = \frac{7}{710} < 0{,}01,	\\[10pt]
|a_{875} - c| = \frac{7}{7006} < 0{,}001.
\end{array}
$$

% ---------------------------- Problem 2----------------------------------
\subsubsection*{\center Задача № 2.}
{\bf Условие.~}
Вычислить пределы функций
$$
\begin{array}{cc}
\textbf{(а):} & \lim\limits_{x\rightarrow1}\frac{x^3-3x+2}{x^3-x^2-x+1}, \\[10pt]
\textbf{(б):} & \lim\limits_{x\rightarrow\infty}\frac{3+2x^2-5x^3}{3x-4x^3+\sqrt[3]{x^6+1}}, \\[10pt]
\textbf{(в):} & \lim\limits_{x\rightarrow0}\frac{3}{1+\sqrt{2x} - (1+x)}{x}, \\[10pt]
\textbf{(г):} & \lim\limits_{x\rightarrow1}\biggl(3-2x\biggr)^{tg\frac{x{\pi}}{2}}, \\[10pt]
\textbf{(д):} & \lim\limits_{x\rightarrow\infinity}\biggl(\tg\frac{4}{x}\biggl(x+4\biggl)\biggl)^\arctg(x^2), \\[10pt]
\textbf{(е):} & \lim\limits_{x\rightarrow0}\frac{2^x-1}{\ln(1-2x)}.
\end{array}
$$
{\bf Решение.~}\\
\text{\bf(а):}
$$
\begin{array}{l}
 \lim\limits_{x\rightarrow1}\frac{x^3-3x+2}{x^3-x^2-x+1} = 
\lim\limits_{x\rightarrow1}\frac{1-1+2}{1-1-1+1} = \biggl[\frac{0}{0}\biggl] = 
\lim\limits_{x\rightarrow1}\frac{(x-1)^2(x+2)}{(x+1)(x-1)^2} = \lim\limits_{x\rightarrow1}\frac{x+2}{x+1} = \frac32.
\end{array}
$$	
\text{\bf(б):}
$$
\begin{array}{l}
\lim\limits_{x\rightarrow\infty}\frac{3+2x^2-5x^3}{3x-4x^3+\sqrt[3]{x^6+1}} = 
\lim\limits_{x\rightarrow\infty}\frac{x^3\biggl(\frac{3}{x^3}+\frac{2}{x}-5\biggl)}{x^3\biggl(\frac{3}{x^2}-4+\sqrt{1+\frac{1}{x^6}}\biggl)} = \frac54
\end{array}
$$	
\text{\bf(в):}
$$
\begin{array}{l}
\lim\limits_{x\rightarrow0}\frac{3}{1+\sqrt{2x} - (1+x)}{x} = \biggl[\frac{0}{0}\biggl] = \lim\limits_{x\rightarrow0}\frac{\frac{2x}{3}-x}{x} = \frac23 - 1 = -\frac13.
\end{array}
$$
\text{\bf(г):}	
$$
\begin{array}{l}
\lim\limits_{x\rightarrow1}\biggl(3-2x\biggr)^{tg\frac{x{\pi}}{2}} = \lim\limits_{x\rightarrow1}(3-2x)^{\tg\frac{{\pi}x}{2}} = e^{\lim\limits_{x\rightarrow1}\tg{\frac{\pix}{2}}(2-2x)} = e^0

\end{array}
$$
\text{\bf(д):}
$$
 \lim\limits_{x\rightarrow\infty}\biggl(\tg\frac{4}{x}\biggl(x+4\biggl)\biggl)^{\arctg(x^2)} = \biggl|
\begin{array}{ll}
y = \frac{1}{x} 	\\ 
{y\rightarrow0}
\end{array}
\biggr| = 4\lim\limits_{y\rightarrow0}(1+4y)^{\frac{1}{y^2}} = 4e^{\lim\limits_{y\rightarrow0}\frac{4}{y}} = 4.
$$
\text{\bf(е):}
$$
\begin{array}{l}
\lim\limits_{x\rightarrow0}\frac{2^x-1}{\ln(1-2x)} = \lim\limits_{x\rightarrow0}\frac{x\ln2}{-2x} = -\frac{\ln2}{2}.
\end{array}
\newpage
$$


% ---------------------------- Problem 3----------------------------------
\subsubsection*{\center Задача № 3.}
{\bf Условие.~}\\
\text{\bf(а):} Показать, что данные функции
$f(x)$ и $g(x)$ являются бесконечно малыми или бесконечно большими
при указанном стремлении аргумента. \\
\text{\bf(б):} Для каждой функции $f(x)$ и $g(x)$ записать главную часть
(эквивалентную ей функцию)  вида $C(x-x_0)^{\alpha}$ при $x\rightarrow x_0$ или $Cx^{\alpha}$
при $x\rightarrow\infty$, указать их порядки малости (роста). \\
\text{\bf(в):} Сравнить функции $f(x)$ и $g(x)$ при указанном стремлении.
\begin{center}
	\begin{tabular}{|c|c|c|}
		\hline
		№ варианта & функции $f(x)$ и $g(x)$ & стремление \\[6pt]
		%\hline
		25 & $f(x) = 2x\arctg\frac{1}{\sqrt{x^4+3}},~g(x)=\sqrt{x}\ln\biggl(\frac{x+2}{x+10}\biggl)$ & $x\rightarrow+\infty$ \\
		\hline
	\end{tabular}
\end{center}
{\bf Решение.~}\\
\textbf{(а):}~Покажем, что $f(x)$ и $g(x)$ бесконечно малые функции,
$$
\begin{array}{cc}
\lim\limits_{x\rightarrow+\infty}f(x) =  \lim\limits_{x\rightarrow+\infty}2x\arctg\frac{1}{x^2} =
\lim\limits_{x\rightarrow+\infty}\frac{2}{x} = 0.\\
\lim\limits_{x\rightarrow\infty}g(x) = \lim\limits_{x\rightarrow\infty}\sqrt{x}\ln\biggl(\frac{x+2}{x+10}\biggl) = 
\lim\limits_{x\rightarrow\infty}\sqrt{x}\ln\biggl(\frac{x}{x}\biggl) = 0.
\end{array}
$$	
\text{\bf(б):}~Так как $f(x)$ и $g(x)$ бесконечно малые функции, то эквивалентными им будут функции вида 
$Cx^{\alpha}$ при $x\rightarrow\infty$. Найдём эквивалентную для $f(x)$ из условия
$$
\lim\limits_{x\rightarrow+\infty}\dfrac{f(x)}{x^{\alpha}} = С,
$$
где $C$ --- некоторая константа. Рассмотрим предел
$$
\lim\limits_{x\rightarrow\infty}\dfrac{f(x)}{x^{\alpha}} = \lim\limits_{x\rightarrow+\infty}2x\arctg\frac{\frac{1}{x^2}}{x^\alpha} = 
\lim\limits_{x\rightarrow\infty}\frac{\frac{2}{x}}{x^\alpha}.
$$
При $\alpha=-1$ последний предел равен $2$, отсюда $C=2$ и 
$$
f(x)\sim \frac{2}{x}~\text{при}~x\rightarrow+\infty.
$$
Аналогично, рассмотрим предел
$$
\lim\limits_{x\rightarrow\infty}\dfrac{g(x)}{x^{\alpha}} = 
\lim\limits_{x\rightarrow\infty}\frac{\sqrt{x}\ln\biggl(\frac{x+2}{x+10}\biggl)}{x^\alpha} = \frac{\sqrt{x}\ln\biggl(\frac{x}{x}\biggl)}{x^\alpha}.
$$
При $\alpha=\frac12$ последний предел равен $1$, отсюда $C=1$ и
$$
g(x)\sim \sqrt{x}~\text{при}~x\rightarrow+\infty.
$$
\text{\bf(в):}~Для сравнения функций $f(x)$ и $g(x)$ рассмотрим предел их отношения при указанном стремлении
$$
\lim\limits_{x\rightarrow\infty}\frac{f(x)}{g(x)}.
$$
Применим эквивалентности, определенные в пункте (б), получим
$$
\lim\limits_{x\rightarrow\infty}\frac{f(x)}{g(x)} = 
\lim\limits_{x\rightarrow\infty}\frac{\frac{2}{x}}{\sqrt{x}} = 
\lim\limits_{x\rightarrow\infty} \frac{2}{x\sqrt{x}} = 0.  
$$
Отсюда, $f(x)$ есть бесконечно малая функция более низкого порядка роста, чем $g(x)$.

% ---------------------------- Problem 4----------------------------------
\subsubsection*{\center Задача № 4.}
{\bf Условие.~}\\
Найти точки разрыва функции 
$$
y = f(x) \equiv 
\left\{
\begin{array}{rcl}
e^{-\frac{1}{x^2}},   &\quad |x|\leq1\\
\frac{1}{\sqrt[3]{x+1}}, &\quad |x|>1\\
\end{array}
\right.
$$
 
и определить их характер. Построить фрагменты графика функции в окрестности каждой точки разрыва. \\
{\bf Решение.~}	
Особыми точками являются точки $x=0,\,-1,\, 1$. Рассмотрим односторонние пределы в окресности каждой из особых точек
$$
\begin{array}{ll}
\lim\limits_{x\rightarrow 1-} e^{-\frac{1}{x^2}} = \lim\limits_{x\rightarrow 1+} e^{-\frac{1}{x^2}}, \\
\lim\limits_{x\rightarrow -1-} \frac{1}{\sqrt[3]{x+1}} = -\infty,\\ 
\lim\limits_{x\rightarrow -1+} \frac{1}{\sqrt[3]{x+1}} = +\infty, \\
\lim\limits_{x\rightarrow 1-} e^{-\frac{1}{x^2}} = e^{-1}, \\
\lim\limits_{x\rightarrow 1+} \frac{1}{\sqrt[3]{x+1}} = \frac{1}{\sqrt[3]{2}}.\\  
\end{array}
$$
\begin{center}
	\begin{tikzpicture}
	
	\begin{axis}[xmin=-2,
	xmax=2, 
	ymin=-2,
	ymax=2,
	width=\textwidth,
	height=0.75\textwidth,
	axis x line=middle,
	axis y line=middle, 
	every axis x label/.style={at={(current axis.right of origin)},anchor=west},
	every inner x axis line/.append style={|-latex'},
	every inner y axis line/.append style={|-latex'},
	minor tick num=1,			
	axis equal=true,
	xlabel=$x$, 
	ylabel=$y$,          
	samples=600,
	clip=true,
	]
	\addplot[blue, domain=-1:1] {e^(-1/(x^2))};
	\addplot[blue, domain=1:3] {1/(x+1)^(1/3)};
	\addplot[blue, domain=-3:-1] {1/(1+x)^(1/3)};
	\end{axis}
	\end{tikzpicture}
\end{center}
Отсюда, точка $x = 1$ --- точка устранимого разрыва 1--го рода, а точка $x = -1$ --- точка
неустранимого разрыва 2--го рода.
  
\end{flushleft}


\newpage
\begin{flushleft}
  \textbf{\LARGE Список литературы}
  \begin{enumerate}
\setcounter{enumi}{0}
\item Львовский С. М. Набор и верстка в системе {\LaTeX}, 2003 с.
\item Чебарыков М. С. Основы работы в системе {\LaTeX}.
\end{enumerate}
\end{flushleft}
\end{document}